\chapter{
Mining Safety Background
\label{chap:2}}

In the thesis by Moore \cite{11124/16423}, the author outlines history of mining tools
dating back to the 1600s. This is when blasting was introduced to the mining process.
Hand tools dominated until the invention of steam powered machines in the mid 1800s.
Around 100 years later, at the time of publication of the thesis, compressed air hammer drills
were the most popular implement. These techniques and the mining drill bits that they used
evolved with each other over time.

The introduction of the continuous mining machine increased productivity and 
was generally regarded as safe since the operator was enclosed in a shielded cab.
Around the 1980s, the remote control for the continuous mining machine led to 
further increases in productivity, but with differing results for safety \cite{KONONOV199521}.
The remote control provides the operator greater visibility of the operation, 
but at increased risk of injury from the machine or tunnel collapse.
The crush hazards from the machine were not mitigated until the introduction of the
personnel proximity detector, around the turn of the millennium.
The detector prevents the machine from advancing into 
workers which are wearing the device \cite{schiffbauer2001active}.

Safety in underground coal mines has not improved over that last decade.
Statistics from 2011 to 2022, shown in \ref{tab:deaths},
show that coal mining has remained a dangerous industry,
especially for underground coal mine operators.
Mine technology has been the subject of extensive review 
\cite{ijerph19042334, 7784796, RALSTON2014305, molaei:hal-02940030}.
Different areas of research include process control and hazard detection.
Process control improvements can avoid hazardous conditions from being created,
making them proactive rather than reactive.

\begin{table}[]
\centering
\caption{Selected statistics from \url{cdc.gov/NIOSH-Mining/MMWC/}. Accessed April 14th, 2024}
\label{tab:deaths}
\begin{tabular}{|l|l|l|l|l|}
\hline
Year & Coal Injury Rate & All Injury Rate & Underground Coal Mine & All Mining Deaths \\
     & (Per 100 FTE)    & (Per 100 FTE)   & Operator Deaths       & (Coal included)   \\ \hline
2011 & 2.49             & 1.94            & 9                     & 37                \\ \hline
2012 & 2.40             & 1.86            & 12                    & 35                \\ \hline
2013 & 2.42             & 1.82            & 14                    & 41                \\ \hline
2014 & 2.48             & 1.83            & 10                    & 43                \\ \hline
2015 & 2.33             & 1.70            & 8                     & 26                \\ \hline
2016 & 2.40             & 1.64            & 7                     & 24                \\ \hline
2017 & 2.56             & 1.64            & 8                     & 27                \\ \hline
2018 & 2.22             & 1.52            & 6                     & 27                \\ \hline
2019 & 2.23             & 1.53            & 7                     & 27                \\ \hline
2020 & 2.30             & 1.42            & 3                     & 29                \\ \hline
2021 & 2.37             & 1.49            & 6                     & 37                \\ \hline
2022 & 2.24             & 1.42            & 6                     & 29                \\ \hline
\end{tabular}
\end{table}

In the cited reviews, examples of sensors in mining include
gas detectors, tunnel mapping, personnel detection, material composition measurement and more.
In the work by Ralston \cite{RALSTON2014305}, goals for autonomous mining systems are outlined in stages
from local manual control to full automation.
Current remote controls allow operators more mobility during cutting, and addition of 
more process feedback can allow them to be further from the cutting interface.
The sensor developed in this work could allow operators to perform tele-operation or tele-supervision,
whereby they are providing higher level commands to the machine using remote sensors for feedback, 
possibly augmented with feedback collected with their own senses.

Previous efforts to improve safety and efficiency have tried automating the continuous mining process,
but fell short on being able to totally take over for human operators \cite{11124/170545, schiffbauer1988testbed}. 
Experienced operators are an invaluable resource. 
They have expert knowledge gained through years of working in difficult conditions,
frequently putting themselves in hazardous locations to collect the feedback they need.
Augmenting their expertise with objective feedback and allowing them to stay at a safer distance
stands to improve outcomes beyond fully human or fully autonomous control.

When working in an underground coal mine, machine operators must stand close to the cutting interface
and the machine to collect feedback to operate the machine. This places them at risk of being crushed 
by the machine or otherwise injured due to hazardous tunnel conditions.
Operators with long term careers are inevitably injured from the loud noises and harmful dust in the environment.
As long as this type of operation remains economically viable, it is likely to continue.

Improving both safety and efficiency is the optimal path for new technologies in an underground mine \cite{Sider1983}.
Mining safety has improved gradually over time and could currently 
be improved by moving operators further from the cutting interface.
Efficiency can be improved by implementing autonomous process control and feedback mechanisms
for optimizing tool change scheduling and cutting control.
To enable these outcomes, we seek feedback mechanisms which can determine quantities of interest for the process.
The next chapter discusses sensor technologies that were considered for 
determining tool wear, material type, and cutting forces in underground coal mining.

