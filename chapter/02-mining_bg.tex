\chapter{
Mining Safety Background
\label{chapTwo}}

In the thesis by Moore \cite{11124/16423}, the author outlines history of mining tools
dating back to the 1600s. This is when blasting was introduced to the mining process.
Hand tools dominated until the invention of steam powered machines in the mid 1800s.
Around 100 years later, at the time of publication of the thesis, compressed air hammer drills
were the most popular implement. These techniques and the mining drill bits that they used
evolved with each other over time.

The introduction of the continous mining machine increased productivity and 
were generally regarded as safe since the operator was enclosed in a shielded cab.
Around the 1980s, the remote control for the continous mining machine led to 
further increases in productivity. 
The remote control provides the operator greater visibility of the operation, 
but at increased risk of injury from the machine or tunnel collapse.
The crush hazards from the machine were not mitigated until the introduction of the
personelle proximity detector, which prevents the machine from advancing into 
workers which are wearing the device.

Safety in underground coal mines has not improved over that last decade.
Other technologies that have been introduced tend to focus on measuring
indicators for collapse after the rock has been damaged such as ...
These technologies are reactionary rather than preventative.
To prevent hazardous situations from developing, rock cutting must be done with care.

Previous efforts to improve efficiency have tried automating the continous mining process,
but fell short on being able to totally take over for human operators. 
Experienced operators are an invaluable resource. They have expert knowledge gained
through years of working in difficult conditions. 
Augmenting their expertise with objective feedback and machine precision stands to 
improve outcomes beyond fully human or fully autonomous control.

Mining safety is really bad and could be improved by moving operators further from the cutting interface.
Efficiency can be improved by implementing autonomous process control and feedback mechanisms.
To enable these outcomes, we seek feedback mechanisms which can determine quantities of interest for the process.
The next chapter discusses sensor technologies that were considered for underground mining.