\chapter{Conclusions and Future Work
\label{chap:11}}

% Introduction: Big picture issues

Underground mine safety can only be improved by the addition of technology which increases both safety and 
productivity. By enabling real time measurement of rock cutting forces, the sensor proposed in this work
could improve safety and productivity by allowing machine operators to make decisions in real time without delaying
the machine operation. This will result in more efficient tool use and lead to reduced time lost from accidents.

Technology is used to improve every aspect of mine operation when possible, evidenced by the large body of 
research across many domains of mine operation \cite{LIU202412}.
Much of the current research focusses on wireless transmission and localization systems \cite{9536741, RayChowdhury2021}.
Other novel technology to improve miner safety and productivity even extends to smart helmets \cite{Sharma2018},
used to augment miner senses with real time information from on board sensors.

The ultimate aim of this research was to produce a force sensor capable of measuring rock cutting forces. 
This was achieved using a capacitive load cell with mostly linear dynamics. It was necessary to iterate the design
a few times to find a configuration that was linear enough. The sensor model was still aided by using nonlinear modeling.
The sensor linearity could be improved by optimizing the film thickness and the number of electrodes.
A very promising technology in the underground mine sector is chipless radio frequency identification, RFID, which uses changes
in capacitance and environmental factors to alter the resonant frequency of a passive antenna \cite{10081221}.
This type of technology would integrate well with the sensor proposed in this work to provide a very low-power solution.

It has long been known that rock cutting forces could be used to infer rock material type or tool wear levels 
if the material and tool condition was known. 
One of the primary contributions of this work is validation of the hypothesis that tool wear and rock type can be 
classified using objective differences in vibrational frequency excitation. 
This work shows that the forces themselves are not strictly needed to make this type of classification. 
Different rock types and tool wear conditions will excite characteristic frequencies in the cutting machinery and environment.
This type of analysis has previously been applied to measurement while drilling, MWD, sytems \cite{LIU202412}, 
but not for force measurements directly from the conical pick. 

The \textit{a-priori} estimation of cutting forces from tool parameters and rock properties has been 
investigated extensively, as an estimate of the expected cutting force is needed for initial tool design \cite{Li2018ATM, Morshedlou2023}.
Other efforts to measure forces for conical picks have used custom blocks with strain gauges \cite{s23239521}, 
or bulky experimental equipment in laboratory conditions \cite{WANG2018210}.
This work is the first to use capacitive load cells for force sensing for conical picks. 
These types of load cells are robust to outside interference, can tolerate large forces,
require very little power to operate, and can be made for very low cost compared to the rest of the tooling and other solutions.
Tools lose mass and efficiency as they wear \cite{Pawlik2022-cj}, so accurate tracking of forces during cutting requires a direct sensor.

Broader impacts of this technology are increased safety for underground miners 
and advancement of rock cutting load cell technology.
Miner safety would be improved by allowing them to perform their role from a greater distance.
Rock cutting load cell technology has been advanced by this study, 
as no prior load cells using capacitive technology have been published for conical picks.
These innovations are the result of years of methodical developments across many domains.

\subsection{Summary of Methods}

This work has contributed to the modeling and use cases for capacitive load cells. 
In Chapter \ref{chap:P3}, equations for the
closed gap sensor are developed, which show that the sensor sensitivity approaches infinity as the force increases.
This relationship works well in the rock cutting domain, as the presence of large forces is interesting
in both rock type and tool wear classification. 
This relationship offsets the insensitivity of the tough sensor design, 
allowing a robust and linear sensor to be sensitive as well.
The dynamics of a capacitive sensor in this application will 
naturally boost these peaks in force while reducing the sensitivity to smaller forces. This makes
detection of the large forces easier.

Based on the results in this work from acoustic tool wear classification, shown in Ch.~\ref{chap:P2}, 
and the tool wear and rock material type classification, shown in Ch.~\ref{chap:P1}, high frequency classification 
should also be considered a valid method for determining material type and tool wear. 
The original goal of this research was to develop a load cell capable of measuring cutting forces.
The first design did not meet this requirement, but it was found that the recorded measurements'
frequency content still varied according to material type and tool wear conditions.
The initial design was not adequately modeled by linear components, but still 
had a consistent response in the frequency domain for different conditions.
This type of sensor is a dynamic sensor, which means that its response
is largely influenced by its dynamics. For the first sensor design,
these dynamics filter and compress the vibrations while still providing enough information for classification.

The sensor in this work was designed to fit a force input range of 0 to 200 kN.
Over this range, it was desired to have at least 10 discrete values, such as increments of 20 kN.
The bandwidth was limited by the capacitance to digital converter, 
which read each channel at a rate of around 400 Hz. The nominal resonant frequency of the sensor 
was around 2 MHz. The max strain for the design was set to 10\% of the length, this was to
minimize fatigue on the case and to preserve linear model assumptions.

\subsection{Summary of Results}

It worked!

\subsection{Discussion}

This work investigated using frequency data for classification in rock cutting,
which proved to be a functional means of identifying both tool wear and material type.

Using the cutting tool to protect the sensor while it measures vibrations or cutting forces,
the sensor was placed behind the sleeve for a balance of protecting the sensing element 
and providing proximity to the cutting forces to be measured.
Integrating the sensor with the conical pick would facilitate easy replacement of the sensor when the pick is replaced,
but likely cause more sensors to be consumed due to the greater proximity to the cutting interface.
Optimal placement of the sensor will require future testing and validation.

\subsubsection{Recommendations}

A simple way to improve the sensitivity of this sensor design would be to stack alternating electrodes and polyimide layers.
Each additional layer increases the sensitivity by increasing the capacitance and reducing the overall stiffness.
A suitable layer height for good deformation characteristics could be stacked to produce a squishier, but more sensitive sensor.

Directly embedding a piezo element within the tool could
also serve to measure higher frequency vibrations which vary with cutting parameters.
Another use for piezo elements in the sensor would be local power generation for the sensor. 
A sensor which could generate its own power from the large forces present would be able to 
eliminate additional wiring from the design. Sensors that generate their own power would be able to 
emit signal wirelessly using the power generated by on-board piezo crystals.

To reduce costs for this design, one could integrate the steel case with the body of the cutting tool.
It would be possible to create a channel for the sensing element in either the block or the sleeve, and then
when the cutting system is assembled, these two components would act as an integrated case for the sensor.
This type of design might increase overall shipping costs if the heavy tools must be sent around and modified before final use.
For this reason, a separate sensor which is lightweight and easy to integrate on-site is desired.

A drawback of the presented sensor design is the use of many integrated electronics. These devices are low cost and low power,
but the margins for the sensing device are small when it comes to cost and power. A sensor design which is mostly analogue but
still capable of wireless transmission of information would be more useful than one which requires additional power
and translation equipment. Such a sensor would be the lowest cost design, as it would be a simple design.
Refining the design to this point will take additional development, but is possible.

\subsection{Conclusion}

The impact of the sensor is limited by the distance of its transmission. 
The further the information can reach, the more productivity can be increased for an underground mine.
If the measurements are incorporated to the machine, the machine could have automatic emergency brakes 
implemented to reduce accidental roof cutting. If the measurements are transmitted to operators, the operators
can infer tool wear without stopping operation and also estimate material strength in real time.
If the measurements are sent to an outside office, trends can be analyzed to determine when machine operators
are the most efficient. Future work should make efforts to streamline the data pipeline starting at the sensor.


