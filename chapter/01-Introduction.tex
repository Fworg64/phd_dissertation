\chapter{Introduction}\label{chap:intro}

The future of underground coal mining is automated. 
Robotic mining systems will determine optimal work patterns and execute
the planned operations with precision.
Progress towards this horizon is gradual in the mining industry.
Improvements to safety in underground coal mining has been sparse recently,
and the National Institute of Ocupational Safety and Health (NIOSH) has sponsored
programs like this one to accelerate the next industrial revolution.
This work focuses on the development of techniques and sensors to automate detection
of material type, tool wear, and cutting forces.

Machine operators in underground coal mines are routinely exposed to dark, dusty, noisy, and
hazardous conditions. When doing their job of operating the continous mining machine,
they must stand close enough to the cutting interface to infer where they are cutting and 
if their tools are worn and need repleaced. They must also stand far enoungh away to avoid
the dangerous rock cutting process. 
A well lit and clean view of a continous miner is shown in \ref{fig:conminer}.
 The long term exposure to hazardous noise and dust causes 
occupational health issues for miners which work for more than several years.

\begin{figure}[ht]
\centering
\includegraphics[width=0.7\textwidth]{Continuous_Miner.jpg}
\caption{Continous Miner. Pictured is the `Remote Continuous Miner HM21 Joy Used for underground coal mining',
 originally uploaded to Wikipedia by user Xlxgoggaxlx 
 under the Creative Commons Attribution-Share Alike 3.0 Unported license
 The machine operator must stand close to the machine and cutting interface to control the machine.
The operators use many cues, but ultimately must track if they are cutting the target material and 
the state of tool wear. Allowing operators to maintain a greater distance while giving them
the feedback they require can improve production and safety.
Image Source/License: \url{https://commons.wikimedia.org/wiki/File:Continuous_Miner.jpg}
}
\label{fig:conminer}
\end{figure}

The technologies proposed in this dissertation could be used to automate feedback collection
during the rock cutting process, allowing operators to perform their duties from a safer location.
This work investigates using measured frequency responses to predict material and wear conditions
using a non-linear dynamic capacitive load cell. 
An acoustic tool wear detection method is also developed using this same premise.
The capstone of this project is a custom capacitive load cell that is able to measure 
rock cutting forces using a linear model. 
Each of these three works are respectively summarized in the included journal articles.

This work was sponsored by NIOSH contract 75D30119C05413.
It is a continuation of prior work at Colorado School of Mines \cite{11124/170545}.
The previous work employed pieozo electric sensors to make measurements to estimate
rock cutting parameters. 
Our work focuses on use of a capacitive based load cell.
Study of rock fracture mechanics \cite{11124/14359} 
and the effects of tool geometry during rock cutting \cite{11124/13192, 11124/16423, 11124/176345} 
have long been pursued to increase mine efficiency and safety.
By providing an \textit{in-situ} force sensor, this work provides a means
for direct measurement of rock cutting parameters.
In addition to enabling operators to perform their roles from a distance,
this device could also be used to optimize tool geometries 
by providing quick and direct feedback of the cutting forces.

The next chapter covers additional background information regarding sensor choice and modelling.
The journal articles that summarize the work are included next.
After the articles, a chapter which compares the different capacitive sensor implentations is given.
An additional chapter describing extensions of the acoustic processing methods is given after that.
The journal articles, and this dissertation, have been published as Open Access 
and are released into the public domain.
If you would like implementation advice, you can contact me, the author, at:
austinfoltmanns at gmail dot com with the subject line containing "THESIS".
I hope you enjoy the rest of this dissertation.

