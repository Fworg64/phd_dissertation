\chapter{Method Selection}

Motivated by the selection of thin film polyimide sensors, 
models for their deformation were studied for this work.
The deformation of polyimide is linear for thick films
and small deformations, but for thin films, 
on the mesoscale of hundreds of micrometers,
polyimide films have nonlinear deformation.
These characteristics are manefested as creep, hystersis, 
and temperature dependence.

Thin film deformation can be modelled using viscous elements such as
\hlre{Maxwell... add figure}.
The system overall is nonlinear, and with a material this viscous,
cycling the film rapidly will result in changes to the vibrational modes.

Other items which govern the vibrational forces during cutting include
the material type and the tool wear. 
Different rock types have different fracture mechanics, 
and these differences can be measured and classified.
Different tool wear levels require different amounts of cutting force,
and will result in changing fracture mechanics depedent of the tool's geometry.

All of these changes in vibrational modes and forces can be tracked in the
frequency domain as changes to the emitted spectra. For this research,
it is known that energy is moving from certain vibrational
modes to other vibrational modes across the categories of material type
and tool wear. This shift in energy over the spectra can be measured
by using the magnitude of the Fourier transform coefficients for a 
short duration sample.

As the Fourier spectra magnitude coefficients change, these changes 
can be used to classify the conditions that caused them.
We use data-driven classification techniques to identify these differences and 
predict the cutting conditions.
Doing so with a capacitive sensor and an acoustic sensor
is discussed in \ref{chap:P1} and \ref{chap:P2}, respectively.
Additional transforms and variations of the technique are discussed in \ref{chap:9}.

A linear sensor for force sensing is also designed for the purposes of this project.
The model of the sensor is derived using regression techniques to develop a linear model 
as well as higher order models.
\ref{chap:P3} discusses the development of the sensor with these models, 
while \ref{chap:8} compares the different capacitive load cell designs for this project.
The last chaper, \ref{chap:10} describes the fit of the linear model and 
further characterization data.

