\chapter{Longer downsampling rates for acoustic classification
\label{chap:9}}

During the study of tool wear classification via acoustic spectra, 
different preprocessing methods were tested. 
The fraction of dimensions that showed significant differences 
across wear categories was used to predict preprocessor performance.
For example, the time domain data does not generally have specific 
time offsets whose sample value correlates to the wear category.
In comparison, the frequency domain data at different frequencies
is likely to change between wear categories due to the changing tool geometry
and increased cutting forces that accompany tool wear.

The results of our study indicated that the unmodified Fourier spectra magnitude values
provided the better classification results than attempting to transform the values further.
For example, the square of these values, a.k.a. the power spectral density, 
was tested as well as the square root of the values. Discussion of these different preprocessing
methods is given below, followed by a discussion of data size in this experiment.


\subsection{Alternate Acoustic Preprocessing Methods}


Plots of data distributions for the wear categories before normalization are shown in \ref{fig:basis}.
Plots of the significant differences between wear categories for each type of preprocessing 
is shown in \ref{fig:basis_compare}. Normalization has no effect on the significance of the differences.
In general, the unmodified Fourier spectra gave the best performance.

Other types of preprocessing tricks were applied in an effort 
to maximize the fraction of dimensions which correlate with tool wear.
Filtering in the frequency domain was applied to smooth 
the frequency response over the frequency domain.
This way, bands of irrelevant frequencies can be brought closer to the
values of neighboring bands that are significant. 
Also, boost filters were applied to increase peaks that are
relevant to wear category and exaggerate the trends.
Ultimately, these techniques did not improve upon the performance
given by the unmodified Fourier spectra magnitude.

The result of low pass filtering in the frequency domain is similar
to the effect of downsampling followed by up-sampling 
via zero padding in the time domain. 
The variance of the 
frequency response over the frequency domain is reduced by clumping
the modes together.
Assuming vibrations are caused by a system that is 
stationary in the short term, the variance of the frequency 
response measurement is reduced by using a longer sample.
This provides increased frequency resolution over the domain,
but at the cost of longer response and classification times.

Window shapes can have subtle but important effects on results.
For the concrete sample in this work, much of the acoustic energy was
below 6 kHz. The higher frequencies still had significant differences,
but the environment in an underground mine dampens these frequencies while
being fairly permissive to lower frequencies. 
Our chosen window captures the lower frequencies, but other windows
could allow better performance from higher frequencies.
Future work should investigate optimal window shapes.

\begin{figure}[t]
\centering
\includegraphics[width=0.85\textwidth]{ch_9_frequencies.png}
\caption{
Data distributions for the tested wear categories using additional Fourier based preprocessing techniques.
The square of the magnitude is an estimate of the power spectral density of the signal.
The square root of the magnitude conditions the signal so that the higher frequencies with small magnitudes
have magnitudes closer to the lower frequencies with large magnitudes.
Neither of these methods gave better classification results than the unmodified Fourier spectra magnitude.
}
\label{fig:basis}
\end{figure}

\begin{figure}[h]
\centering
\includegraphics[width=0.85\textwidth]{ch_9_frequencies_compare.png}
\caption{
Tests for significant differences in frequency content between wear categories for the additional 
Fourier based preprocessing techniques. The square root was able to bring the differences between 
more high frequencies into significance compared to the square, or power spectral density. 
Squaring a signal will make outliers larger, while the square root conditions by bringing all numbers closer to unit value.
Neither of these methods adds more information or improves the signal basis for classification over the spectra magnitude. 
However, they do perform much better compared to using time domain data.
}
\label{fig:basis_compare}
\end{figure}

\subsection{Comments on Data Size}

For this experiment, around 90 seconds of acoustic data was used. 
To ensure sufficient power of the results, small samples had to be used. 
It was found early in the study that long samples with some downsampling gave good results.
The limits of the reliability of this method have not yet been found, but as sample length increases,
recorded signals will capture more information for a reliable estimation of cutting parameters.

More data should be collected to investigate using even longer window length for samples.
This window length is only limited by the necessary rate of classification and acceptable delay.
In general, the classification will be delayed by the length of the window. 
The classification calculations are much shorter in computation time than the sample length, so this
does not add much to the delay.
Longer sample lengths stand to increase classification accuracy, and additional demand for computational
resources in this case can be mitigated by downsampling the recorded signal.

Another thing that large sample windows can help mitigate is undesired performance caused by outside interference
on the audio signal. Performance under these conditions would need to be characterized for a wide number of 
circumstances. One benefit the acoustic sensor has in this regard though is that almost every frequency
bin value is correlated to a tool wear condition. This means that even if some frequencies are
experiencing interference, there are likely many other frequencies that can still be used to make an accurate classification.

