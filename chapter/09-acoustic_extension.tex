\chapter{Longer downsampling rates for acoustic classification
\label{chap:9}}

During the study of tool wear classification via acoustic spectra, 
different preprocessing methods were tested. 
One metric that was used to asess preprocessor performance was
the fraction of dimensions that showed significant differences 
across wear categories.
For example, the time domain data does not generally have specific 
time offsets whose sample value correlates to the wear category.
In comparison, the frequency domain data at different frequencies
is likely to change between wear categories due to the changing tool geometry
and increased cutting forces that accompany tool wear.

Efforts to maximize the fraction of dimensions which correlate with tool wear,
filtering in the frequency domain was applied.
By smoothing the frequency response over the sampled domain,
bands of irrelavent frequencies can be brought closer to the
values of neighboring bands that are significant. 
Also, by boosting trends in the frequency spectra, peaks that are
relevant to wear category can be increased in magnitude.
Examples of these effects are shown in \hlgr{fig}.

The result of low pass filtering in the frequency domain is similar
to the effect of downsampling followed by upsampling 
via zero padding in the time domain. 
The variance of the 
frequency response over the frequency domain is reduced by clumping
the modes together.
If we assume that the vibrations are being caused by a system that is 
stationary in the short term, we can reduce the variance of the freqeuncy 
response measurement by using a longer sample.
This provides increased frequency resolution over the domain,
but at the cost of longer response and classification times.

Window shapes can have subtle but important effects on results.
For the concrete sample in this work, much of the acoustic energy was
below 6 kHz. The higher frequencies still had significant differences,
but the environment in an underground mine dampens these frequencies while
being fairly permissive to lower frequencies. 
Our chosen window captures the lower frequencies, but other windows
could allow better performance from higher frequencies.

