\subsection{Capacitive Sensor Linearity and Implementation Recommendations
\label{chap:10}}

To map the crushed gap sensor measurements to estimated forces, 
a regression to the forces as measured by the 
Linear Cutting Machine's strain gauges was used.
This method has a few limitations, the most significant
being the distance between the tested sensor and the strain gauges.
Because of the distance, the higher frequency force components will
be out of phase when compared. 
In addition to the distance, the viscous nature of the polyimide dielectric distorts the 
frequency response when the two measurements are compared.
The electronics for the sensor also contribute high frequency noise to the measurement.
For these reasons, the higher frequency components of the force are not directly correlated 
between the two systems. 

A plot of the transfer function between the measurements is shown in \ref{fig:sense_tf}.
The regression target has been low pass filtered with a cutoff frequency of 10 Hz,
which can be seen by the steep drop in magnitude at this point.
The individual sensor channels have the same filter applied to limit high frequency input.
The cross spectrum, $P_{yx}$, shows that there is good correlation between the low frequency
components of the estimate and the target.
The transfer function between the estimate and the target suggests that additional 
filtering after the regression method could improve results.
Post-processing requirements depend on the application, and 
additional filtering after regression could be a useful technique for tuning performance.

The final sensor prototypes are shown in \ref{fig:airgap}.
The sensor case provides additional protection from the environment.
The device is assembled via later welding.
Future sensor designs could omit the steel case and integrate the sensor directly within
the block or sleeve of the tool.
This type of sensor measures cutting forces via the change in capacitance caused
by the displacement of the top of the case when force is applied. 

\begin{figure}[ht]
\centering
\includegraphics[width=5.5in]{ch10_tf.png}
\caption{
Power spectral density of sensor, target, and transfer function between them.
}
\label{fig:sense_tf}
\end{figure}

\begin{figure}[ht]
\centering
\includegraphics[width=0.5\textwidth]{ch10_airgap_and_case.jpg}
\caption{
The air gap sensor membrane, left, and an assembled prototype, right.
}
\label{fig:airgap}
\end{figure}

The capacitive steel donut with viscous polyimide filling is a useful base design
for many robotic applications. Robustness, linearity, and sensitivity are important 
parameters for any sensor design. Positive and negative correlations between each 
of these categories and key design paramters are listed in \ref{tab:improve}.
To overcome the conflicting directions between sensitivity and the other design goals,
 use a thicker film and compress it down so that it becomes more thin and stiff.
This type of sensor can be used to measure rock cutting forces, make predictions on tool wear,
and make classifications of rock type during rock cutting.

When looking to increase the sensitivity of the capacitive load cell design, 
most of the design parameters, shown in \ref{tab:improve},
that can be considered, put sensitivity at odds with the other design goals of robustness and linearity.
On the other hand, when looking to increase robustness and linearity, many design parameters 
increase both of these qualities together. Considering these aspects of the designs highlights
the two extremes for sensor designs in this application, sensitive and delicate designs or tough and insensitive designs.
In a rock cutting application, the sensor should be on the tougher side or it stands to be consumed too rapidly,
before it can provide enough useful measurements to offset its cost of inclusion in the operation.

\begin{table}[]
\centering
\caption{Changes to design parameters that would improve certain categories}
\label{tab:improve}
\begin{tabular}{|r|c|c|c|}
\hline
Parameter               & Robustness   & Linearity    & Sensitivity               \\ \hline
Dielectric Thickness    & +            & +            & -   \\ \hline
Dielectric Stiffness    & +            & +            & -   \\ \hline
Dielectric Permittivity & -            & +            & +   \\ \hline
Case Walls Thickness    & +            & +            & -   \\ \hline
Sensing Electrode Area  & +            & +            & -   \\ \hline
\end{tabular}
\end{table}

When it comes to integrating this sensor with the target application,
construction of the sensor contributes to overall linear performance.
The steel case is assembled using the laser weld procedure described in Appendix \ref{app:laser}.
Simulation for the capacitance of the air gap is described in Appendix \ref{app:sim}.
Computer aided solving of the analytical model equations for the air gap and closed area sensors
is shown in Appendix \ref{app:math}. 

When using this design as a template, the key parameters to consider are the stiffness of the case walls,
the thickness of the dielectric material, and the resonant frequency of the sensor across the expected input range.
The stiffness of the sensor case makes the sensor robust, allowing it to handle large input forces.
The thickness of the dielectric determines the fine characteristics of the sensor deformation. 
The resonant frequency of the sensor over the input range determines the bandwidth of the sensor.
By carefully choosing each of these values for the application, a suitable sensor can be designed like 
the one featured in this work.


